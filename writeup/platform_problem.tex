%\documentclass[11pt]{article}

\usepackage{booktabs}
\usepackage{colortbl}
\usepackage{dcolumn} 
\usepackage{epstopdf}
\usepackage{fourier}
\usepackage{fullpage}
\usepackage{graphicx}
\usepackage{hyperref}
\usepackage{longtable} 
\usepackage{natbib}
\usepackage{rotating}
\usepackage{setspace} 
\usepackage{Sweave} 
\usepackage{tabularx}

\hypersetup{
  colorlinks,
  citecolor=blue,
  linkcolor=blue,
  urlcolor=blue,
  filecolor=white
}

\newtheorem{proposition}{Proposition}

\title{Here is a title}

\begin{document}
   \maketitle



 

\section{Platform's design challenge} 
Different platforms have different preferences of the matching of
platform participants. These preferences presumably differ over
time. For example, a nascent platform is probably more worried about
market liquidity and growth, where as an established platform, focuses
more on revenue, cost and retention. It is beyond the scope of this
paper map various macro-level platform preferences to different
strategies implemented via budgets for sellers. However, it is still
useful to (a) characterize a common platform problem and (b) show how
it is addressable via the controlled allocation of prominence.

A common problem in electronic markets where quality is a factor is
that novices---those without a market reputation---have difficulty in
finding a trading partner and/or must sell a discount. For example,
\cite{resnick2006value} found that new sellers without an eBay
reputation ended up selling identical goods at an 8\% discount
compared to veteran seller. This problem is perhaps more acute in
labor markets, where uncertainty about seller quality and reputation
is greater and potentially more important. 

\cite{tervio2009superstars} develops a model of a labor market in
which hiring firms focus too much attention on veterans and not enough
on ``talent discovery'' i.e., the hiring of novices. The gist of the
model is that talent discovered via hiring becomes common knowledge
post-hire, and since workers are not obligated to stay with their
hiring firm, hiring firms cannot fully recoup the costs of talent
discovery. \cite{pallais2010inefficient} tested this model on oDesk
with an experiment in which she hired a treatment group of 952 workers
for a task and gave them public feedback for their work.  The control
group received no job (and hence no feedback). Most of the workers in
the experiment had no experience on the platform. Pallais finds that
her treated group of workers were able to substantially raise their
earnings once equipped with a reputation. In fact, the workers were
able to earn more from other buyers than the total cost of the
experiment with a few months.

The theoretical results from Tervi{\"o} and empirical results from
Pallais imply a clear challenge for platform creators, which is to
push more firms to hire novices. In the model below, we will assume
that the platform simply wants to maximize total social surplus. A
for-profit platform actually has different incentives depending on
price structure (for example, a platform that charges a fixed
percentage will want to maximize the NPV of the wage bill), but for
illustrative purposes, these complications are not relevant.

% Novices---those without feedback or reputations---often have great
% difficulty getting started in markets (TK - cite Zeckhauser). Online
% labor markets are known to suffer this problem to an extreme degree
% (TK - cite Pallais, Tervio, Stanton). In the Tervio model, The root of
% the problem is that talent discovery is inefficiently under-supplied
% in the marketplace, at least in part because workers cannot subsidize
% their own discovery.
 
% \subsection{Match probability as mediated by platform creators}

% \subsection{Price level and structure---carved in stone?} 

% When a platform entirely forgoes prices, prominence is still
% allocated, but the allocation depends on some other criteria set by
% the platform, such as lexical ordering, recent activity, chronology
% etc. 

% For example, craigslist orders results within
% categories by time of posting, as does Amazon Mechanical Turk
% (MTurk).\footnote{Not surprisingly, this time-based ordering leads to
%   manipulation as participants try to increasing their prominence. See
%   \cite{chilton2010task} for a discussion of this phenomena on MTurk.}
% On other sites, returned results are often ordered based on some
% algorithmic notion of relevancy, such as tf-idf or PageRank
% \cite{page1999pagerank}.

\subsection{Platform growth and talent discovery} 
Consider a marketplace with a fixed pool of buyers with measure
$1$. We will assume that the sellers are selling labor and that a
buyer choosing to purchase labor from a particular seller can be
thought of as hiring that seller. There are two types of sellers:
those that are good, and produce an output of $1$ if hired, and bad,
that produce nothing if hired. The proportion of good sellers is
$\theta$ and the measure of sellers if $\theta^{-1}$.

All sellers start as \emph{novices}. When a seller is hired, their
type is revealed publicly to the marketplace. Bad sellers exit the
marketplace, while good workers become \emph{veterans}. We will assume
that all parties are price takers and that wages are $w_V$ for
veterans and $w_N$ for novices, with $\theta < w_N < w_V < 1$ and $1-
w_v > \theta - w_n$. This ensures that firms would rather hire a
veteran than a novice, but they would rather hire a novice that not
hire anyone at all.

For any discount rate, the platform wants (a) all veterans to matched
with buyers, to get the guaranteed payoff this promises) and (b) all
other buyers hiring novices, in order to turn some fraction into
veterans. If matching occurs in continuous time and $v(t)$ is the
number of veterans in the marketplace, the socially optimal evolution
of the marketplace will be:
\begin{equation}
v_{0}'(t) = (1-v_0(t))\theta
\end{equation}
with $v_0(0)=0$, which has the solution of $v_0(t) = 1-e^{-\theta
  t}$. However, if the platform cannot fully control the matches that
are formed, we can have a situation where multiple buyers compete for
the services of the same veteran, even though only one match is
possible. We will assume that veterans pick randomly from the pool of
buyers courting them. Further, we will assume that buyers can sort
themselves such that they can obtain the best possible terms (i.e.,
all veterans will receive the same number of applicants, in contrast
to a ball-and-urn model of applications where the realized number is
binomial random variable).

Let us not assume that a fraction $r$ of buyers will pair with novices
and $1-r$ with veterans. We will indicate the measure of veterans at
time $t$ as $v_1(t)$. This makes $v_1'(t) = r\theta$. At time $t$, the
probability that buyer seeking the services of a veteran will be
successful is $v(t) (1-r)^{-1}$. In equilibrium, the returns to
seeking out a veteran have to be equal to seeking out a novice, hence
$\theta - w_n = v_1(t)(1-r)^{-1} (1 - w_v)$ so $r = 1 - (1 -
  w_v)(\theta - w_n)^{-1}v(t)$, and hence: 
\begin{equation}
  v_1'(t) = \left(1 - \frac{1 - w_v}{\theta - w_n}v_1(t)\right)\theta 
\end{equation} 
Since $1 - w_v > \theta - w_n$, we know that $v_1'(t) < v_0'(t)$---the
supply of veterans grows more slowly in the unorganized market where
buyers are allowed to compete for the scarce pool of veterans. Not
only is growth slower, it eventually stalls completely once $\theta -
w_N = v_1(t)(1-w_v)$. This ``stalling'' of the marketplace occurs once
the number of veterans reaches $\underline{v} = (\theta - w_n)(1
  - w_v)^{-1}$. Once the platform reaches $\underline{v}$, the
talent-discovery stops and no novices every get hired: buyers are
content to take their chances with the over-subscribed veterans. The
expression for $\underline{v}$ offers intuitive comparative statics:
the lower the baseline level of talent, lower the $\theta$, the more
rapidly the marketplace stalls. Higher wages for novices has a
similarly stifling effect, as does low wages for veterans.

\subsection{Tilting the marketplace towards optimal growth}
If the platform wanted to continue to grow at the optimal rate, it
could divert the excess buyer attention away from the veterans and
towards the buyers. To formalize this intuition, let us imagine that
we are allocating the buyer attention (which as the same measure as
the buyers, i.e., $1$) amongst the veterans the novices. We can think
of this allocation as a ``budget'' we give to a veteran $b_v$ and a
budget given to a novice, $b_n$.

To obtain the optimal growth path for the platform, we need to
allocate $v(t)$ of the attention to the veterans and $1-v(t)$ to the
novices. The number of novice sellers at time $t$ is $\theta^{-1} -
v(t)\left(1 - \theta^{-1}\right)$, and so
\begin{eqnarray}
b_v &=& v(t) \nonumber\\
b_n &=& \frac{1-v(t)}{\theta^{-1} - v(t)\left(1 - \theta^{-1}\right)}
\end{eqnarray} 

\subsection{Static example with heterogeneous match probabilities}
To give an example of the usefulness of this prominence budget
conception, let us consider a platform that is no longer growing but
simply wants to maximize the number of matches formed. Unlike in the
simplified example before, let us more realistically assume that each
seller $i$ has some exogenous match probability $q_i$ when exposed to
a single buyer. As we assumed previously, sellers have limited
capacity and thus can form at most one match. This means that for any
seller, there are decreasing returns to attention by buyers. Let $b_i$
be the number of ``impressions'', allocated to a seller. The total
amount of impressions is $A$ (for attention).

The probability that a seller fails to form a match when given one
unit of impressions is $1-q_i$. The probability of forming no matches
when giving $b$ units of visibility is $(1-q_i)^b$ and thus the
probability of forming a match (and nice approximation to this
probability) is:
\begin{equation} 
m_i(b) = 1 - (1 - q_i)^{b} \approx 1 - e^{-q_i b}
\end{equation} 
The platform now faces a constrained optimization problem:  
\begin{equation}
\max_{\vec{b}} \quad L = \sum_i \left(1 - e^{-q_i b_i}\right) + \lambda \left(A - \sum b_i\right) 
\end{equation} 
The first order condition is $q_i e^{-q_i b_i} - \lambda \ge 0$. The
shadow value of the constraint is $\lambda$, which can be thought of
the additional matches possible from one additional unit of attention
to distribute. At the optimum, each seller either (a) offers the same
marginal probability of forming a match or (b) gets no attention at
all. We can re-formulate the FOC as a linear equation $-q_i b_i + \log
\lambda (q_i)^{-1} = 0$ conditional upon $a_i > 0$. 

First let us consider the $n=2$ case and assume that there is an
interior solution, i.e., $b_1 > 0$ and $b_2 > 0$. The first order
conditions are:
\begin{eqnarray}
q_1 e^{-b_1q_1} + \lambda &=& 0 \nonumber\\
q_2 e^{-b_2q_2} + \lambda &=& 0 \nonumber\\
b_1 + b_2 = A 
\end{eqnarray} 
which we can re-write as 
\begin{eqnarray}
-b_1q_1 + b_2q_2  &=& \log \frac{q_2}{q_1} \nonumber\\
b_1 + b_2 & = & A 
\end{eqnarray} 
which is a system of linear equations. To ease explication in the $n >
2$ case, we will write this system in matrix notation. 
\begin{equation}
\left( \begin{array}{cc}
-q_1 & q_2  \\
1 & 1  \end{array} \right) 
\left( \begin{array}{c}
b_1   \\
b_2    \end{array} \right)
=
\left( \begin{array}{c}
\log \frac{q_2}{q_1}   \\
A    \end{array} \right)
\end{equation} 

Let us now consider the $n > 2$ case. The matrix equation is the
standard equation for a system of linear equations: $M \vec{b} =
\Gamma$. We can define the entries of $M$ as:  
\begin{equation}
\mu_{ij} =
\begin{cases}
-q_i, & \text{if }i = j \wedge j \ne n \\
q_{i+1}, & \text{if }i + 1 = j \wedge j \ne n \\ 
1, & \text{else} 
\end{cases}
\end{equation}
Note that the last row contains all $1$'s---this is just the
constraint that the sum of all allocations equals the attention
available. To construct the column vector $\Gamma$, we can use the
same simplification of dividing the $i$th equation by the $i + 1$th
equation to create:
\begin{equation}
\gamma_{i} = 
\begin{cases}
\log \frac{q_{i+1}}{q_{i}}, & \text{if }i < n \\
A, & \text{if }j = n \\ 
\end{cases}
\end{equation}
We have a system of linear equations that we can solve.  
\begin{equation} 
[ m_{ij} ] \vec{b} = [ \gamma_i ]
\end{equation} 

\paragraph{Finding interior solutions}
Not all sellers will get attention in this formulation: if $q_k <
\lambda^*$ (i.e., the marginal benefit from a seller when $a=0$ is
less than the shadow value of the constraint at the optimum), the
seller is excluded. One complication in the linear system approach
outlined above is that solving the system does not exclude these
sellers (they receive negative weight). Further, we cannot simply find
the first negative weight seller and draw the cut-off there: the
cutoff depends on the shadow value estimated with the proper set of
workers. 

A heuristic for solving the problem is to use a binary search to
identify the smallest-q sellers that still receives positive attention
the optimal solution, operating on list of sellers, sorted from
highest to lowest value. This does require solving the system of
linear equations at each step, though the speed of binary search
(combined with some probably places for optimization) does not make
this intractable.  

% k = p_1 e^{p_1 x_1} =  p_2 e^{p_2 x_2} = p_3 e^{p_3 x_3}
% ln(p_1) + p1*x1
% Without coming up with an actual budget for each seller, 

% The platform creator wants to make sure that of the buyer's searches,
% a fraction $v$ ending up on veterans, and a fraction $1-v$ end up on 
% novices.  

%\end{document}


\subsection{Applied allocation of prominence}
In the previous example, we talk about allocating prominence as if it
were something that could be apportioned out by the platform by fiat,
in whatever fine-grained distribution required. In reality, prominence
on a platform is mediated by page position/rank order in returned
search results, with each position offering some flow of page views
and clicks (i.e., prominence). This means that prominence comes in
``lumps'' that differ (often dramatically) in size, with the
distribution of lump sizes seeming to follow familiar patterns across
sites. What this means in practice is that turning platform
preferences into prominence budgets is not sufficient: we need some
mapping from these budgets to actual positions.

Let us assume that we have $n$ individuals that we need to assign to
$n$ slots. Each of the $n$ positions has a ``value'' $v$ (i.e., an
associated visibility) such that $v_1 > v_2 > v_3 \ldots v_n$. In this
paper, we will be concerned solely with stochastic assignment
mechanisms, i.e., a mechanisms that will place an individual $i$ at
position $j$ with probability $p_{ij}$. We evaluate allocations based
on the expected values they generate for an individual $i$, which we
will call $x^n_i = \sum_{j=1}^n p_{ij}v_j$, as well as the
distributional properties of the allocation mechanism.

