
\section{Introduction} 

In recent years, a number of electronic platforms that bring together
buyers an sellers have emerged. Examples include Apple's app store
(software, music, videos etc.), Amazon's seller network, eBay
(physical goods), and labor or expertise (oDesk, Innocentive, Elance,
99 Designs etc.). Prior research on two-sided platforms has focused
primarily on price structure and levels and competition between
platforms \citep{armstrong2006competition, rochet2003platform,
  rochet2006two, parker2005two}. This literature is largely descriptive, but it could
thought of as falling in the ``design economics'' literature
\citep{roth2002economist}, where price structure and levels are the
main instruments.

% While it 
% It is easy to show that under realistic assumptions that even under
% the guidance of a benevolent social planner, the 

%  The
% reason for this design focus is that the platform needs to impose some
% kind a price structure and level and these choices have implications
% for revenue.\footnote{The problem is analogous to the enormous
%   literature in public economics on optimal taxation}. We will show,
% using a simple but empirically grounded model that even if the
% platform was a benevolent social planner and had no need to impose
% fees of any kind, the ``natural'' outcome of interactions between
% buyers and sellers might not be welfare enhancing.

Despite the existing platform literature on price, we take the view
that price level and structure are not the only---and probably not he
most useful---instruments in the platform's ``toolkit.'' The reason is
that changing price structure and levels proves difficult in practice
once a marketplace has gained traction. Further, price is difficult to
experiment with, particularly when participants can easily compare
notes. \footnote{The recent Netflix pricing debacle is an example of
  the difficulty in making substantive price structure and level
  changes.} Furthermore, even if prices were easy to change, having a
simple price structure is often viewed as per se attractive from a
marketing perspective.

Given the limitations of price as a tool for platform design, what
other factors can platforms control that can shape platform
interactions in ways beneficial to the platform? We argue that one of
the most ``tune-able'' and powerful instruments in the platform
toolkit is the control of \emph{prominence} of users to each
other. Prominence directly affects the probability of matches forming,
since obviously parties that never see each other can never trade.

In traditional, non-electronic markets, buyers and sellers are
responsible for making themselves known to their would-be trading
partners. Gaining prominence might require as little effort as showing
up to trade at some specific time and place, or it might require a
massive advertising campaign. Regardless, the market's allocation of
prominence is de-centralized, arising from the decisions of many
individual actors.\footnote{There seems to be no consensus on the
  welfare properties of advertising. \cite{dixit1978advertising} argue
  that under a range of plausible assumptions, advertising is socially
  inefficient, though there are other works that take alternate views,
  claiming it depends on a number of factors
  \citep{becker1993simple}.}  By comparison, in electronic markets and
on electronic platforms, prominence is inherently determined by the
platform creators, as what is seen by users is set by the design
choices and policies of the platform. This power to control prominence
has not gone unnoticed or unexploited: the multi-billion dollar
position-based online advertising industry
\citep{varian2007position,edelman2005internet} is evidence for this
point.

Given the enormous literature on position auctions, one might be
tempted to think that the allocation of prominence is a solved
problem, with the solution being auctions. Although many platforms do
make use of auctions, they are far from universal and even when they
are used, often coexist with the unpaid ``organic'' display of
participants. There are numerous reasons why a platform might eschew
auctions but the general explanation is that platform member
preferences and willingness to pay do not perfectly align with
platform's preferences: the platform might worry that those choosing
to pay for position might be adversely selected, or the platform might
want to secure some minimal level of prominence for all users, to
ensure their continued participation and thus sufficient market
liquidity.  Even the existing of an explicit pay-for-prominence scheme
might undermine the credibility and thus the usefulness of the
platform as a whole.\footnote{An historic example is the legal
  prohibition on ``Payola,'' or the practice of record companies
  paying DJ's to play certain songs.} 

% TK transaction costs of auctions 

% We can now describe 

\subsection{Overview of the paper} 
In this paper, we consider the problem on a platform creator that has
some preference over the matches formed on the platform and that these
preferences will be acted upon by the platform allocating individuals
some ``share'' of prominence in the marketplace. To motivate the
problem, we first present a very simple model of platform dynamics
that will show how a situation in which all participants have full
knowledge all participants (i.e., prominence is irrelevant) is not
socially optimal. We will show that this situation can be improved by
restricting prominence. In the model, market failure arises from too
much attention being focused on platform ``veteran'' sellers, with too
little discovery of novice sellers---a problem known to be important
in several marketplaces.

The focus of the paper will shift to solving the problem in the form
it is most likely to take in practice---allocating individuals to some
number of ``slots,'' each which afford the occupant of that slot some
share the of ``prominence'' which is attention from would-be buyers
and can be thought of in practice as page views or impressions. We
will assume that attention comes from buyers and that sellers are the
ones being allocated prominence, though in reality, search could be
conducted by buyers, sellers or both. Platform preferences for seller
prominence are modeled as a ``budget'' for each seller, which is the
share of the available prominence that the seller should receive in
expectation. We will assume that the attention itself is homogeneous,
i.e., we will not draw distinctions between which buyer is generating
the attention.

We will show by example that seemingly reasonable strategies for
allocating sellers to slots (such as always rank ordering sellers by
some notion of merit like feedback scores) leads to highly unequal
allocations of prominence that are likely to be far away from the
platform's preferences. Furthermore, these simple allocation
mechanisms lack desirable properties like continuity (which can cause
sellers with only slight differences receiving radically different
amounts of attention on the platform). Using data from an actual
two-sided platform, where we have measures of prominence and
congestion, we will show that the buyer attention to different
``slots'' (position in search results) follow a Zipfian distribution
and that unless the platform's preferences for the allocation of
prominence to sellers also follows the precise Zipfian distribution of
attention shown to the slots.

In this framework, the allocation of individuals to slots becomes an
economic assignment problem. We review the existing literature on
these mechanisms and propose two mechanisms of our own. The first
mechanism is guaranteed to give allocations of prominence consistent
with the platform's preferences (as captured by the sellers
budgets). It works by solving an linear program. Unfortunately, it is
impractical, having a computational complexity of $\mathcal{O}(n^5)$,
making in unsuitable for the large-scale two-sided platforms
motivating the problem.

As an alternative, we propose a substitute which we call the ``reverse
tontine'' (RT) mechanism. While it does not guarantee that prominence
allocations are consistent with seller budgets, it does offer
continuity and monotonicity in budgets with respect to the expected
prominence allocation. Furthermore, the distribution of prominence
allocations under RT for a higher budget seller first-order
stochastically dominates the outcomes for a lower budget seller,
ensuring that a rational seller will always prefer to have a higher
budget. Simulation studies using parameters from an online labor
market show that the RT does quite well in implementing the platform's
preferences. As a practical matter, we show that RT allocations can be
generated very quickly and has limited memory requirements. It is also
easy to update with changed budgets and changes in the prominence
associated with different slots, making it suitable for application on
platforms with large numbers of participants.


