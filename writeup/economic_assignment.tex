\section{The economic approach to allocation} 
The allocation of visibility on a platform has clear analogies to the
economic allocation problem, in that the platform has a number of
slots that differ in their value and a number of items---people,
websites, classifieds, etc.---that must be assigned to a slot. The
economic literature on making these kinds of allocations in the
absence of a price mechanism goes back to
\cite{hylland1979efficient}. In the setting Hylland-Zeckhauser
formulation, individuals have cardinal preferences, which they use to
``buy'' probability shares, giving a bistochastic matrix which can be
used to generate allocations.

Subsequent work on the assignment problem has focused on cases where
only ordinal preferences are known. The two main solutions in the
literature are the random serial dictatorship and the probabilistic
serial mechanism. In the serial dictatorship, each person is serially
chosen at random and gets to make their preferred choice from the
remaining choices. In the probabilistic serial mechanism proposed by
\cite{bogomolnaia2001new}, every individual simultaneously ``eats''
unit probability shares at a constant, common rate, with each
individual starting at their preferred position. When an individual's
current position is totally consumed, they move to their next best
option until time runs out. \cite{budish2009implementing} gives a
procedure for drawing allocations from the resultant bistochastic
matrix. A growing literature in economics explores the properties of
these allocation mechanisms, including their efficiency and
strategy-proofness (or lack thereof)
\cite{kojima2010incentives,manea2007serial,abdulkadiroglu1999house}.

While the economic assignment problem provides a framework, the
motivation for this paper---the allocation of visibility on two-sided
platforms---has properties that make the standard economic approaches
unattractive. One difference is that the economic literature is
generally focused on cases where all individuals should be treated
equally, whereas platforms almost always want to give more visibility
to some select participants.

As we explore the notion of treating different participants
differently, it will be useful to have a more formal statement of the
problem. Let us assume that we have $n$ individuals that we need to
assign to $n$ slots. Each individuals has ``merit'' $b$, creating a
vector of budgets, $\vec{b}$.  Each of the $n$ positions has a
``value'' $v$ (i.e., an associated visibility) such that $v_1 > v_2 >
v_3 \ldots v_n$. In this paper, we will be concerned solely with
stochastic assignment mechanisms, i.e., a mechanisms that will place
an individual $i$ at position $j$ with probability $p_{ij}$. We
evaluate allocations based on the expected values they generate for an
individual $i$, which we will call $x^n_i = \sum_{j=1}^n p_{ij}v_j$,
as well as the distributional properties of the allocation
mechanism. This formalization immediately highlights another important
difference between this setting the common economic allocation
setting: notice that the values associated with positions are not
conditional, i.e., we are assuming that all individuals have identical
ordinal preferences, with everyone preferring more visibility to
less. As such, the serial dictatorship and the probabilistic
dictatorship give identical results.\footnote{To see why, consider
  that in the probabilistic serial mechanism case, all $n$ individuals
  would start ``eating'' probability at the most valued position
  (since this is every individual's first choice). Each would be able
  to eat $1/n$ and then, all at once, all would move to position 2,
  where they would eat $1/n$, and so on. The resultant bistochastic
  matrix is identical to that generated by the serial dictatorship.}

Another important difference between the classic economic allocation
scenario and our setting is that the economic literature was motivated
by one-off scenarios such as the assigment of students to schools,
families to houses, kidneys to patients, etc. As such, there is a
natural focus on the equity of the single realized allocation. In
contrast, on platforms individuals can be re-allocated many hundreds
of times per day, appearing in different orders in response to
different queries. As such, expected visibility and the properties of
the distributio of realized outcomes becomes more interesting and more
important.  A final difference that is common to many (but not all)
two-sided platforms is that the limited capacity of the platform
participants suggests that frequent changes in ``merit'' should be
made, so as to make sure the same over-exposed individuals do not
continually appear at the top of the search results.

% \subsection{Practical importance of justice}
% Within two-sided platforms, the ``thing'' being allocated visibility
% is often not a website or a product but a person.  This is a
% non-trivial difference, as it gives a technical resource allocation
% problem an ethical dimension and as such, any choice by the platform
% creator is likely to generate strong emotions. Whether they perceive
% it that way or not, the platform creator is implicitely assigning
% visibility according to some notion of justice. This is likely to
% prove contentious. Aristotle wrote in Book V of his Nichomachean
% Ethics:
% \begin{quote}
%   [T]his is the origin of quarrels and complaints---when either equals
%   have and are awarded unequal shares, or unequals equal
%   shares. Further, this is plain from the fact that awards should be
%   ``according to merit''; for all men agree that what is just in
%   distribution must be according to merit in some sense, though they
%   do not all specify the same sort of merit...
% \end{quote} 
% We can easily achieve Aristotle's ideal when goods are divisible, but
% the problem becomes more challenging when goods (or ``bads'') are
% indivisible, e.g., jobs, unpleasant duties, artwork, custody of
% children, etc. An ancient solution to this dilemma is to introduce some
% randomization device like drawing lots.

% Pure randomization works well when all should be treated equally, but
% what if individuals are unequal? It then no longer seems fair to
% allocate by a lottery that treats individuals exactly the same (a
% point made by Aristotle). 

One attractive approach is to employ a simple serial
dictatorship. Individuals are ranked by merit; the first-ranked
individual gets their first choice, the second gets to pick from what
is remaining, and so on until all choices are exhausted. This is
actually the only mechanism that is fair and efficient, in the sense
that a higher-ranked individual would never envy a lower-ranked
individual \cite{balinski1999tale}.

In a one-shot scenario, the simple serial dictatorship is an appealing
solution, but if done repeatedly, with the same set of choices and the
same set of individuals, it can easily run afoul of Aristotle's theory
that equals should be treated equally, as well as his argument that
rewards should be proportional to merit.

To see the problem, suppose there are two individuals with budgets
$b_1 = 0.8$ and $b_2 = 0.2$ that must be assigned to two positions
that offer value (i.e., visibility) worth $v_1 = 80$ and $v_2 =
20$. The simple serial dictatorship is the only obviously fair
solution: the higher merit individual gets the higher value
position. Further, the benefits are proportional to relative
merit. However, as either the ratio of merits or ratio of position
values change, the obviousness fairness of the simple serial
dictatorship wanes: the higher merit person still gets the higher
value position, but proportionality is lost. And no matter how close
the two individuals become in merit, gap between their expected payoff
does not diminish. Even when $b_1 = 1 + \epsilon$ and $b_2 = 1$,
individual $1$ gets a (proportionally speaking) windfall of $30$ and
individual $2$ loses $50$, no matter how small $\epsilon$.

Intuitively, as $\epsilon$ gets smaller, the two individuals should
each get $v_1$ and $v_2$ in approximately equal proportion. This of
course means that sometimes the strictly lower-merit individual gets
the top position, sacrificing fairness in one realization for a gain
of fairness in expectation. We want the mechanism to capture the
notion that likes should be treated alike, which in this case means
that expected value is continuous in merit. It practically goes
without saying, but expected value should also be monotonically
non-decreasing in merit (i.e., an individual should never wish they
had a smaller budget).

% \subsection{Rivalrousness and the need for frequent re-allocation} 
% TK - Make much shorter. 

% In situations where search is returning information (e.g., websites,
% documents, etc.), there is no inherent problem with returning the same
% result to every user in response to the same query. This relatively
% static approach (with updates only made to reflect improved estimates
% of relevancy) works because the returned results are non-rivalrous
% goods: the effect of the marginal visitor on the remaining visitors is
% effectively zero, which means that every visitor can be shown the
% ``best'' page without that page being consumed. While this
% non-rivalrousness is true of web pages, it is not true for many other
% things that might be returned from search, such as individual buyers
% and sellers.

% Rivalrousness becomes an issue when the returned results have limited
% capacity, which is characteristic of many two-sided platforms (e.g.,
% most individuals can only work 40 to 50 hours per week, go on so many
% dates, knit so many custom sweaters, etc.). The issue is that
% visibility in search tends to follow a power law distribution, with
% the top results getting a large fraction of the views and hence
% contacts by would-be buyers or sellers. The fire hose of attention can
% quickly exhaust the capacity of those on top. In short order, the the
% individuals getting the most visibility need it the least and
% searchers are quickly discouraged by the low response rate to their
% contacts. This characteristic makes it critical for applied purposes
% that budgets can be updated in nearly real-time.

