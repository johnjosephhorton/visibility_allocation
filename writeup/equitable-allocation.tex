
\section{Allocation mechanisms}
The allocation of prominence on a platform has clear analogies to the
economic allocation problem, in that the platform has a number of
slots that differ in their value and a number of items---people,
websites, advertisements, etc.---that must be assigned to a slot. The
economic literature on making these kinds of allocations in the
absence of a price mechanism goes back to
\cite{hylland1979efficient}. In the setting Hylland-Zeckhauser
formulation, individuals have cardinal preferences, which they use to
``buy'' probability shares, giving a bistochastic matrix which can be
used to generate allocations.

Subsequent work on the assignment problem has focused on cases where
only ordinal preferences are known, with the literature presenting
various mechanisms and evaluating their efficiency and
strategy-proofness (or lack thereof)
\citep{kojima2010incentives,manea2007serial,abdulkadiroglu1999house}. The
two main solutions in the literature are the serial dictatorship and
the probabilistic serial mechanism.\footnote{In the probabilistic
  serial mechanism proposed by \cite{bogomolnaia2001new}, every
  individual simultaneously ``eats'' unit probability shares at a
  constant, common rate, with each individual starting at their
  preferred position. When an individual's current position is totally
  consumed, they move to their next best option until time runs
  out. \cite{budish2009implementing} gives a procedure for drawing
  allocations from the resultant bistochastic matrix.}  In the serial
dictatorship, individuals are ranked according to some criterion
(possibly randomly ranked); the first-ranked individual gets their
first choice, the second gets to pick from what is remaining, and so
on until all choices are exhausted. This is actually the only
mechanism that is fair and efficient, in the sense that a
higher-ranked individual would never envy a lower-ranked individual
\cite{balinski1999tale}.

While the economic assignment problem provides a framework, allocating
prominence on platforms has a number of important distinguishing
features. One important difference is that on a platform, we expect to
re-allocate individuals to slots continually, perhaps even in response
to every new query by a buyer. In contrast, the economic literature
was motivated by one-off scenarios such as the assigment of students
to schools, families to houses, kidneys to patients, etc. As such,
there is a natural focus in economics on the equity of the single
realized allocation. In the platform case, we are must more interested
in expectations and the distributional properties of assignment
mechanisms.

In a one-shot ``ecnonomic'' scenario, the simple serial dictatorship
is an appealing solution, but if done repeatedly, with the same set of
choices and the same set of individuals, it can easily run afoul of
the common-sense notion that equals should be treated equally. To see
the problem, suppose there are two individuals with budgets $b_1 =
0.8$ and $b_2 = 0.2$ that must be assigned to two positions that offer
value (i.e., visibility) worth $v_1 = 80$ and $v_2 = 20$. The simple
serial dictatorship is the only obviously fair solution: the higher
budget individual gets the higher value position. Further, the
benefits are proportional to relative budgets. However, as either the
ratio of budgets or ratio of position values change, the obviousness
fairness of the simple serial dictatorship wanes: the higher budget
person still gets the higher value position, but proportionality is
lost. And no matter how close the two individuals become in budgets,
gap between their expected payoff does not diminish. Even when $b_1 =
1 + \epsilon$ and $b_2 = 1$, individual $1$ gets a (proportionally
speaking) windfall of $30$ and individual $2$ loses $50$, no matter
how small $\epsilon$.

Intuitively, as $\epsilon$ gets smaller, the two individuals should
each get $v_1$ and $v_2$ in approximately equal proportion. This of
course means that sometimes the strictly lower-merit individual gets
the top position, sacrificing fairness in one realization for a gain
of fairness in expectation. We want the mechanism to capture the
notion that likes should be treated alike, which in this case means
that expected value is continuous in merit. Further, expected value
should also be monotonically non-decreasing in merit (i.e., an
individual should never wish they had a smaller budget).

% the
% motivation for this paper---the allocation of prominence on two-sided
% platforms---has properties that make the standard economic approaches
% unattractive. One difference is that the economic literature is
% generally focused on cases where all individuals should be treated
% equally, whereas platforms almost always want to give more visibility
% to some select participants.


% Each individuals has ``merit'' $b$, creating a
% vector of budgets, $\vec{b}$.  

% This formalization immediately highlights another important
% difference between this setting the common economic allocation
% setting: notice that the values associated with positions are not
% conditional, i.e., we are assuming that all individuals have identical
% ordinal preferences, with everyone preferring more visibility to
% less. As such, the serial dictatorship and the probabilistic
% dictatorship give identical results.\footnote{To see why, consider
%   that in the probabilistic serial mechanism case, all $n$ individuals
%   would start ``eating'' probability at the most valued position
%   (since this is every individual's first choice). Each would be able
%   to eat $1/n$ and then, all at once, all would move to position 2,
%   where they would eat $1/n$, and so on. The resultant bistochastic
%   matrix is identical to that generated by the serial dictatorship.}

% Another important difference between the classic economic allocation
% scenario and our setting is that the economic literature was motivated
% by one-off scenarios such as the assigment of students to schools,
% families to houses, kidneys to patients, etc. As such, there is a
% natural focus on the equity of the single realized allocation. In
% contrast, on platforms individuals can be re-allocated many hundreds
% of times per day, appearing in different orders in response to
% different queries. As such, expected visibility and the properties of
% the distributio of realized outcomes becomes more interesting and more
% important.  A final difference that is common to many (but not all)
% two-sided platforms is that the limited capacity of the platform
% participants suggests that frequent changes in ``merit'' should be
% made, so as to make sure the same over-exposed individuals do not
% continually appear at the top of the search results.

% \subsection{Practical importance of justice}
% Within two-sided platforms, the ``thing'' being allocated visibility
% is often not a website or a product but a person.  This is a
% non-trivial difference, as it gives a technical resource allocation
% problem an ethical dimension and as such, any choice by the platform
% creator is likely to generate strong emotions. Whether they perceive
% it that way or not, the platform creator is implicitely assigning
% visibility according to some notion of justice. This is likely to
% prove contentious. Aristotle wrote in Book V of his Nichomachean
% Ethics:
% \begin{quote}
%   [T]his is the origin of quarrels and complaints---when either equals
%   have and are awarded unequal shares, or unequals equal
%   shares. Further, this is plain from the fact that awards should be
%   ``according to merit''; for all men agree that what is just in
%   distribution must be according to merit in some sense, though they
%   do not all specify the same sort of merit...
% \end{quote} 
% We can easily achieve Aristotle's ideal when goods are divisible, but
% the problem becomes more challenging when goods (or ``bads'') are
% indivisible, e.g., jobs, unpleasant duties, artwork, custody of
% children, etc. An ancient solution to this dilemma is to introduce some
% randomization device like drawing lots.

% Pure randomization works well when all should be treated equally, but
% what if individuals are unequal? It then no longer seems fair to
% allocate by a lottery that treats individuals exactly the same (a
% point made by Aristotle). 


% \subsection{Rivalrousness and the need for frequent re-allocation} 
% TK - Make much shorter. 

% In situations where search is returning information (e.g., websites,
% documents, etc.), there is no inherent problem with returning the same
% result to every user in response to the same query. This relatively
% static approach (with updates only made to reflect improved estimates
% of relevancy) works because the returned results are non-rivalrous
% goods: the effect of the marginal visitor on the remaining visitors is
% effectively zero, which means that every visitor can be shown the
% ``best'' page without that page being consumed. While this
% non-rivalrousness is true of web pages, it is not true for many other
% things that might be returned from search, such as individual buyers
% and sellers.

% Rivalrousness becomes an issue when the returned results have limited
% capacity, which is characteristic of many two-sided platforms (e.g.,
% most individuals can only work 40 to 50 hours per week, go on so many
% dates, knit so many custom sweaters, etc.). The issue is that
% visibility in search tends to follow a power law distribution, with
% the top results getting a large fraction of the views and hence
% contacts by would-be buyers or sellers. The fire hose of attention can
% quickly exhaust the capacity of those on top. In short order, the the
% individuals getting the most visibility need it the least and
% searchers are quickly discouraged by the low response rate to their
% contacts. This characteristic makes it critical for applied purposes
% that budgets can be updated in nearly real-time.

\subsection{Exact solution} 
In this section we show how to determine a collection of $n\times n$
permutation matrices $\Pi_{k}$, together with a collection of
probabilities $\alpha_{k}$, such that the marginal probabilities
$p_{ij}$ induced by the $\Pi_{k}$'s and the $\alpha_{k}$'s (that is,
$p_{ij}=\sum_{k}\alpha_{k}\pi_{ij}^{k}$, where $\pi_{ij}^{k}$ denotes
the $ij$th element of the matrix $\Pi_{k}$) satisfy the exact
proportion criterion \[
\frac{b_{i}}{\sum_{j}b_{j}}=\frac{\sum_{j}p_{ij}v_{j}}{\sum_{j}v_{j}}\,\]
If no such collection exists (which we can also verify), we give a
collection of permutation matrices whose ``distance'' from the exact
proportion criterion is minimized in the least-squares sense, i.e. we
find marginal probabilities $p_{ij}$ that minimize\[
\sum_{i}\left\Vert
  \frac{b_{i}}{\sum_{j}b_{j}}-\frac{\sum_{j}p_{ij}v_{j}}{\sum_{j}v_{j}}\right\Vert
^{2}\,\] To simplify notation, we suppose without loss of generality
that $\sum_{i}b_{i}=\sum_{j}v_{j}=1$ and therefore our exact
proportion condition is merely that $b_{i}=\sum_{j}p_{ij}v_{j}$ for
all $i$. It is obvious that the matrix of marginal probabilities, $P$,
should be the solution to the least-squares problem\[
\mathrm{minimize}_{P}\left\Vert Pv-b\right\Vert _{2}^{2}\] which is
easily solved using standard convex optimization methods (if desired
one could also consider minimizing $\left\Vert Pv-b\right\Vert _{1}$
or $\left\Vert Pv-b\right\Vert _{\infty}$, which can be done using
linear programming). The optimal objective function value to the above
problem is $0$ precisely when $P$ satisfies the criterion.

Let $P^{*}$ denote the optimal solution to the least-squares problem;
we now consider the problem of determining a collection of permutation
matrices $\Pi_{1},\dots,\Pi_{K}$ and positive scalars
$\alpha_{1},\dots,\alpha_{K}$ such that the marginal probabilities
induced by the $\Pi_{k}$'s are precisely the matrix $P^{*}$, i.e. that
$p_{ij}^{*}=\sum_{k}\alpha_{k}\pi_{ij}^{k}$ for all $i,j$. This is a
straightforward application of a theorem of Birkhoff
\cite{marcus}. The main idea is that if we are given an $n\times n$
doubly stochastic matrix $P$, there must exist an $n\times n$
permutation matrix $\Pi$ such that $\pi_{ij}=1$ implies that
$p_{ij}>0$ (i.e. $\Pi$ only has $1$'s in entries whose corresponding
value in $P$ is nonzero). If this is the case, we let $t$ be the
minimum value of $P$ whose corresponding entry in $\Pi$ is $1$,
i.e. \[ t=\min_{i,j:\pi_{ij}=1}p_{ij}\,\] Clearly if $t=1$ then $P$ is
a permutation matrix. If $t<1$ then the matrix $(1-t)^{-1}(P-t\Pi)$
has one fewer nonzero entry than $P$. We then perform this procedure
recursively on the elements of $P$ (at most $(n-1)^{2}$ times) until
$t=1$ and we are done; see Algorithm \ref{alg:find-equitable}.  Since
Algorithm \ref{alg:find-equitable} requires that we solve a bipartite
matching problem $\mathcal{O}(n^2)$ times, and since the Hungarian
algorithm requires $\mathcal{O}(n^3)$ running time
\citep{burkard2009assignment}, the overall complexity of Algorithm
\ref{alg:find-equitable} is $\mathcal{O}(n^5)$.

\begin{algorithm}
\caption{Exact Proportion (EP). This algorithm takes as input a pair
  of nonnegative $n$-vectors $b$ and $v$ such that
  $\sum_{i}b_{i}=\sum_{j}v_{j}=1$ and outputs an $n\times n$ marginal
  probability matrix $P^{*}$ that minimizes $\left\Vert
  Pv-b\right\Vert _{2}^{2}$, as well as a collection
  $\mathtt{Permutations}$ of at most $(n-1)^{2}$ permutation matrices
  $\Pi_{k}$ and associated probabilities $\alpha_{k}$ such that
  $P^{*}$ is the marginal probability matrix of the $\Pi_{k}$'s and
  the $\alpha_{k}$'s, i.e. $p_{ij}^{*}=\sum_{k}\alpha_{k}\pi_{ij}^{k}$
  for all $i,j$.  }
\label{alg:find-equitable}
\begin{algorithmic}
\STATE Let $P^{*}$ be the probability matrix that minimizes $\left\Vert Pv-b\right\Vert _{2}^{2}$.
\STATE \COMMENT{This can be found quickly using the method of least-squares.}
\STATE Set $\bar{P}:=P^{*}$ and $\mathtt{Permutations}:=\emptyset$.
\WHILE{$\bar{P}$ has more than $n$ nonzero entries}
\STATE Let $Q$ be defined by $q_{ij}=\left\lceil p_{ij}\right\rceil $ (i.e.
$Q$ has a $1$ in it whenever the corresponding entry of $P$ is
nonzero).
\STATE Solve a maximum-weight bipartite matching problem using the Hungarian
algorithm \cite{burkard2009assignment} with the matrix $Q$ as an
input. Let $\Pi$ denote the output assignment matrix.
\STATE \COMMENT{By \cite{marcus} it must be the case that $\pi_{ij}=1$
implies that $p_{ij}>0$ for all $i,j$.}
\STATE Set $\mathtt{Permutations}:=\mathtt{Permutations}\cup\Pi$.
\STATE Set $t=\min_{i,j:\pi_{ij}=1}p_{ij}$.
\STATE Set $\bar{P}:=(1-t)^{-1}(\bar{P}-t\Pi)$.
\STATE \COMMENT{By \cite{marcus} it must be the case that this reduces the
number of nonzero entries of $\bar{P}$ by at least one.}
\ENDWHILE
\STATE \COMMENT{At this point we now know that $P^{*}$ is a convex combination
of the matrices $\Pi$ in $\mathtt{Permutations}$; we can merely
recover the weighting of the matrices using convex optimization.}
\STATE Let $K:=\left|\mathtt{Permutations}\right|$. Let $\alpha_{1},\dots,\alpha_{K}$
be the solution to the convex optimization problem\begin{eqnarray*}
\mathrm{minimize}_{\alpha_{1},\dots,\alpha_{K}}\left\Vert (\alpha_{1}\Pi_{1}+\cdots+\alpha_{K}\Pi_{K})-P^{*}\right\Vert _{2}^{2} &  & s.t.\\
\sum_{k}\alpha_{k} & = & 1\\
\alpha_{k} & \geq & 0\,\forall k\,.\end{eqnarray*}
\RETURN $P^{*}$, $\mathtt{Permutations}$, and $\left\{ \alpha_{1},\dots,\alpha_{K}\right\} $.
\end{algorithmic}
\end{algorithm}

\subsection{Approximate solution}
Algorithm \ref{alg:find-equitable}, or the Exact Proportion (EP)
algorithm shows that meeting the exact proportion criterion is
possible in some situations and gives a method of constructing the
necessary bistochastic matrix. Further, by meeting the equity
criterion, it gives us continuity in merit. However, in the platform
applications that motivated the problem, this approach is likely to
prove infeasible: we cannot solve an LP every time an individual's
budget changes or even make draws from the bistochastic matrix.

An algorithm we can use in applications should at least offer
continuity and monotonicity in merit, and ideally equity. It should be
easy to incorporate changes in $\vec{b}$ or $\vec{v}$ and deliver
assignments as they are needed, in the order that they are likely to
be needed, i.e., we can learn the first position before the second
position and the second position before the third, and so on.

One other desirable feature for the algorithm---a feature not usually
seen as an important attribute---is conceptual simplicity, as it will
make it easier for the individuals subject to the control of the
algorithm to understand and hopefully support as just the resultant
allocation. Distributive justice, which is what the algorithm is
attempting to deliver, is only one kind of justice---there is also
procedural justice \cite{rawls1999theory}, which is concerned with the
transparency and fairness of the process. Being able to clearly
explain to platform participants why they are appearing in search in
way they can understand would be a potentially powerful advantage.

\label{sec:tontine}
We propose a simple approach we call the ``Reverse Tontine" (RT) that
is nearly identical to the simple serial dictatorship, except that the
dictator $i$ at each step is chosen with probability
$\frac{b_i}{\sum_j b_j}$, where $j$ indexes all the unallocated
individuals in that step. The name was chosen because in this
mechanism, it is desirable to ``die'' early (unlike an actual
tontine), and when you die, your probability shares are distributed to
``survivors.''

\begin{algorithm} 
\caption{Reverse Tontine (RT). This algorithm takes as input vectors
  $\vec{b}$ of individual budgets shares and $\vec{v}$ of values. Both
  vectors are of the same length. The algorithm returns an assignment
  of individuals to values.}
\label{alg:tontine}

\begin{algorithmic}
\WHILE{there are unassigned individuals} 
\STATE Draw an individual $i$ with probability equal to their budget share,
$b_i \left(\sum_j b_j\right)^{-1}$. 
\STATE Assign individual $i$ to position $\max \vec
v$. 
\STATE Remove the drawn individual from $\vec b$ and the selected position
from $\vec v$ 
\ENDWHILE
\end{algorithmic}
\end{algorithm} 


Given the description of the Algorithm~\ref{alg:tontine}, one might
worry that it would require $n-k$ re-normalizations of the budgets at
step $k$, which combined with the steps needed to actually make the
draws might make this approach computational
unattractive. Fortunately, there is a simple and fast algorithm that
requires no re-weighting and can generate draws with $n$ individuals
in $\log_2 n$ steps. To do this, we first make the $n$ individuals
the terminal nodes in a binary tree. We combine pairs of nodes, giving
the parent node a value equal to the sum of the values of its
children. Let $T$ be our resultant binary tree, which consists of
nodes $t \in T$. The right and left children of a non-terminal node
$t$ are $r(t)$ and $l(t)$, respectively. Each node has a value, $|t|$:
if the node is terminal, then $|t_j| = b_j$, otherwise $|t_j| =
|l(t_j)| + |r(T_j)|$. If a node is missing children, its value is just
the value of its remaining child (if any).

To sample an individual, we start at the root node, $t_0$. It has a
value equal to the sum of all individual budgets. We draw a random
number $\gamma \sim U[0,1]$. If $\gamma < \frac{|l(t_0)|}{|l(t_0)| +
  |r(t_1)|}$, we move to the left child node; otherwise we move to the
right. We then repeat the same procedure, using the values of the new
nodes children. Once we reach a terminal node, we have the individual
assigned to that position. We complete the process by moving back up
the tree, ``repairing'' the values for each tree to reflect that the
selected individual is no longer available for allocation.

\begin{proposition}
The binary tree draw method selects individuals proportional to their
budget share.
\end{proposition} 
\begin{proof}  
Let $b_i$ be the budget of a particular individual. They are selected
iff a unique series of moves down the tree occurs, $m_1,m_2,\ldots
m_k$ that generate a sequence of visited nodes $t_0t_1,t_2,\ldots t_k$
(where $t_0$ is the root). The probability of this sequence occurring
(and thus $b_i$ being selected) is:
\begin{eqnarray*} 
  Pr(b_i) = \frac{|t_1|}{|l(t_0)| + |r(t_0)|} \times
  \frac{|t_2|}{|l(t_1)| + |r(t_1)|} \times \ldots \times
  \frac{b_i}{|l(t_{k-1})| + |r(t_{k-1})|}
\end{eqnarray*} 
Because $|t_k| = |r(t_k)| + |l(t_k)|$, we can cancel all numerators
and denominators except for the first denominator and the last
numerator. This gives us $Pr(b_i) = \frac{b_i}{|l(t_0)| + |r(t_0)|}$,
and since $|t_0| = \sum_j b_j$, $Pr(b_i) = \frac{b_i}{\sum_j b_j}$.
\end{proof} 

%\begin{prop} 
%The computational complexity is $O(n \log n)$ and the memory
%complexity is $O(n)$.
%\end{prop} 
%\begin{proof}
%  TK.
%\end{proof} 

\begin{proposition}
An individual's expected outcome from the RT mechanism is continuous
in their budget share.
\end{proposition}
\begin{proof} 
  
  Let us first define notation for the expected value of an individual
  with budget $b_i$ when there are $n$ other individuals as $x_n(b_i,
  \vec{b}, \vec{v})$. We can proceed by induction. First, when $n=2$,
  under the RT mechanism, with $b_1 > b_2$ and $v_1 > v_2$, we have
  $x_2(b_1, \vec{b}, \vec{v}) = b_1v_1 + b_2v_2$ and $x_2(b_2,
  \vec{b}, \vec{v}) = b_1v_1 + b_2v_2$. Both expected values are
  linear functions of the respective budgets and hence
  continuous. Now, we assume that $x_m(.)$ is continuous and consider
  the $m+1$ case in which we add a new position $v'$ and a new
  individual with budget $b'$.

With a probability equal to his or her budget share, $b_i$ is selected
immediately and receives $v^* = \max v \cup v'$. Let us denote that
probability as $s_{m+1}(i) = \frac{b_i}{\sum_{b \in \vec{b} \cup b'}
  b}$, which is continuous in $b_i$. If the individual $i$ does not
get selected in that round, some other individual $k$ is selected,
with probability $s_{m+1}(k)$. When that $k$ individual is selected,
they remove $v^*$ from consideration and remove their budget,
$b_k$. For each of the $k$ scenarios this generates, this leaves the
individual $i$ facing an $m$-sized RT scenario. We can write the
expected value recursively, as:


\begin{equation*} 
x_{m+1}(b_i, \vec{b} \cup b', \vec{v} \cup v') = 
s_{m+1}(i) v^* + 
\left(1- s_{m+1}(i) \right) 
\sum_{k=1}^{m} s_{m+1}(k) x_m (b_i, \vec{b} \cup b' \setminus b_k, \vec{v} \cup v' \setminus v^*) 
\end{equation*}


We can see that $x_{m+1}()$ is a finite linear combination of
continuous functions (recall that $x_{m}(.)$ is continuous by
assumption) and hence is itself continuous.
\end{proof} 

\begin{proposition} 
  The reverse tontine mechanism does not always satisfy the equity
  criterion.
\end{proposition} 
\begin{proof}
We can show this with a simple counter-example with $n=2$. The
equitable allocation for $b_1 > b_2$ (with $b_1 + b_2 = 1$ and $v_1 >
v_2$ is $\mathbf{E}[u_1] = b_1 \left(v_1 + v_2\right)$ while the
weighted serial approach gives: $\mathbf{E}[u_1] = b_1 v_1 + b_2
v_2$. For the two expectations to be equal, $b_1 = b_2$.
\end{proof} 

\begin{proposition} \label{prop:fosd}
The distribution of outcomes for an individual first order
stochastically dominates (FOSD) the distribution of outcomes for any
other individual with a lower budget, facing the same collection of
other individuals.
\end{proposition}
% http://en.wikipedia.org/wiki/Stochastic_dominance
\begin{proof}
  Assume that we have two individuals with budgets $b_h$ and $b_l$
  respectively, with $b_h > b_l$. We want to compare the
  distributions of the two individuals under RT. First, let us define
  $S_m(b)$ as the expected sum of the budgets for the individuals
  after $m$ individuals have already been selected, not including
  individual with budget $b$. Choose any position $k$ such that $1 \le
  k < n$.  We are interested in the probability that the individual
  received a position $q \le k$ (and hence of higher value), which we
  can think of as a cdf, $Pr(q \le k) = F(k; b)$. We can write the
  probability of an individual being selected after $k$ as a function
  of their budgets. For $b_h$, we have: 
  \begin{equation*}
    1 - F(k; b_l) = \left(1-b_l\right)\left(1-\frac{b_l}{S_1(b_l)}\right)
    \ldots \left(1 - \frac{b_l}{S_k(b_l)} \right) 
    \end{equation*} 
whereas for $b_l$, we have: 
     \begin{equation*}
    1 - F(k; b_h) = \left(1-b_h\right)\left(1-\frac{b_h}{S_1(b_h)}\right)
    \ldots \left(1 - \frac{b_h}{S_k(b_h)} \right) 
    \end{equation*} 
     
Because $b_h > b_l$ and by the assumption that both the high and low
individuals face the same collection of individuals, for all $k$,
$S_k(b_h) = S_k(b_l)$, and hence $(1-b_h S_m(b_h)^{-1}) < (1-b_l
S_m(b)^{-1})$, which means that every term of $1-F(k; b_h)$ is less
than every term of $1-F(k; b_k)$ and thus the distribution of outcomes
for $b_h$ FOSD $b_l$.
\end{proof} 

A direct implication of Proposition \ref{prop:fosd} is that any
utility maximizer, regardless of attitude towards risk, would prefer
to have a higher budget to a lower budget.

