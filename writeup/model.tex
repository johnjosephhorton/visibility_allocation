\documentclass[11pt]{article}

\usepackage{booktabs}
\usepackage{colortbl}
\usepackage{dcolumn} 
\usepackage{epstopdf}
\usepackage{fourier}
\usepackage{fullpage}
\usepackage{graphicx}
\usepackage{hyperref}
\usepackage{longtable} 
\usepackage{natbib}
\usepackage{rotating}
\usepackage{setspace} 
\usepackage{Sweave} 
\usepackage{tabularx}

\hypersetup{
  colorlinks,
  citecolor=blue,
  linkcolor=blue,
  urlcolor=blue,
  filecolor=white
}

\newtheorem{proposition}{Proposition}

\title{Here is a title}

\begin{document}
   \maketitle



 

\section{Platform model}
A platform has constant demand (i.e., buyers) each period, normalized to measure 1. There is also supply (i.e., workers) normalized to measure 1. Buyers are homogeneous, whereas workers have a ``type'': with probability $\theta$, a worker is good, in which case when paired with a buyer, they generate an output of $y$. When they are bad (which happens with probability $1-\theta$), they generate an output of $0$. We will assume that the cost of production for the worker is $0$. The only way the market/platform can learn if a worker is good or bad is by having them be hired by a buyer for one period. No firm will ever hire a worker revealed to be bad, and hence when a rookie is hired and revealed to be bad, the rookie exits the marketplace. If they are good, they become a veteran. 

The platform fully controls matching and decides who buyers hire. Each period, the firm has to decide what fraction of demand is paired with novices. Future periods are discounted with $\delta \in (0,1)$. The total social surplus is thus: 
\begin{equation} 
V(t) = \sum_{\tau=0}^t \left(y(1-x_\tau) + yx_\tau\theta\right)\delta^\tau
\end{equation}
subject to the constraint on the policy function that $\forall t 1-x_t \le s_v(t-1)$, where $s_v(t) = \sum_{\tau = 0}^t \theta x_\tau$, i.e., the number of buyers paired with veterans cannot exceed the number of veterans available in that period, based on past choices of the policy variable. 

\subsection{Questions}

\begin{itemize} 
\item What's the platform's optimal policy function? 
\item Under what conditions will the firm ever leave a veteran unpaired in a period?
\end{itemize} 

\input{insituend.tex}